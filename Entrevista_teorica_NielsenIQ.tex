\documentclass{article}
\usepackage[utf8]{inputenc}

\title{Entrevista teórica NielsenIQ}
\author{joserc17 }
\date{April 2022}

\begin{document}

\maketitle

1. ¿Qué distribución/IDE de Python has usado?


Jupyter, Spyder, Colab y Atom


2. ¿Qué librerías has usado? ¿Cómo cargas esas librerías en un código?


Numpy, Pandas, Seaborn y Matplotlib
import numpy as np
import pandas as pd
import seaborn as sns
import matplotlib.pyplot as plt


3. ¿Conoces una librería para procesar bases muy grandes? ¿Cuáles?


Si, Spark y Dask


4. ¿Python es un lenguaje sensitivo al uso de mayúsculas y minúsculas?


Si, la prueba de esto sería el ejemplo de si hay dos variables llamadas “Var” y 
“var” serían estas dos variables distintas


5. ¿Cuál es la diferencia entre listas y tuplas?


La principal diferencia es que las listas son mutables, es decir se puede editar, 
agregar y borrar información mientras que en las tuplas no se puede hacer nada 
de esto 


6. ¿Qué es un diccionario?


Son un tipo de estructuras de datos que nos permite guardar un conjunto no 
ordenado de pares siendo esta clave-valor


7. ¿Cómo se define una función?


Se define de la siguiente forma:
def my_function (param1, param2, …)
pass
Donde declaramos la función primero poniendo la palabra “def” seguido del 
nombre de la función, para el ejemplo le hemos puesto “my_function” donde 
después en los paréntesis se encuentran los parámetros y por último la palabra 
“pass” que es el contenido de la función 


8. ¿Cómo funciona la función map?


La función map() es para aplicar una función a cada elemento en un diccionario, o 
en una lista (iterables) y devolver otro iterador que podemos usar en otra parte de 
nuestro programa
\end{document}
